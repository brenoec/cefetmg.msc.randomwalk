
\section{Introdução}

A Caminhada Aleatória é uma abstração de um fenômeno físico, simulada em
sistemas computacionais, que possui similaridade no eixo $x$.

\vspace{5mm}
Se construírmos gráficos de caminhadas aleatorias distintas sobre um mesmo
sistema de eixos, poderemos observar a fomação de um envelope, limitado por
valor proporcional a $t^{1/2}$. A raiz da média quadrádica dos pontos gerados
para diferentes caminhadas em determinado passo $t$ se aproximará do limite do
envelope, à medida que aumentarmos o número de caminhadas.

\vspace{5mm}
O expoente do envelope é o expoente de difusão, que é um dos expoentes que
caracterizam a classe de universalidade na qual a Caminhada Aleatória está
inserida.
