
\section{Materiais e Métodos}

As simulações tem teor probabilístico. Logo, se mostrou necessária a
implementação de um gerador de números pseudoaleatórios; portanto. O gerador
implementado foi o \texttt{Ranq2} \cite{Press:Numerical:Recipes}.

\vspace{5mm}
O método generalizado de box-müller \cite{Tsallis:2007:GeneralizedBoxMuller}
foi utilizado para gerar as distribuições Gaussiana e q-Gaussiana. O método
utiliza dois valores pseudoaleatórios gerados uniformemente para gerar um par
de valores na distribuição desejada.

\vspace{5mm}
O Pseudocódigo \ref{alg:rand-gaussian} mostra o procedimento para a geração
de um par através do método, para distribuição Gaussiana. É necessário
substituir a função logaritmo pela função q-logaritmo de $q\prime$ para
gerarmos um par na distribuição q-Gaussiana:

\begin{equation}
  q \in (-\infty, 3), \quad q \neq 1,
\end{equation}

\begin{equation}
  q\prime = \frac{1+q}{3-q},
\end{equation}

\begin{equation}
  x > 0,
\end{equation}

\begin{equation}
  ln_q = \frac{x^{1-q}-1}{1-q}.
\end{equation}

\begin{algorithm}
  \caption{Gerador de números pseudoaleatórios para Gaussiana}
  \label{alg:rand-gaussian}

  \begin{algorithmic}[1]
    \Function{RandomGaussian}{}
    \Statex

      \Let{$n_1$}{\texttt{Ranq2}} \Comment{$n \in [0,1)$}
      \Let{$n_2$}{\texttt{Ranq2}}
      \Statex

      \Let{$z_1$}{$\sqrt{\smash[b]{-2 \:ln(n_1)}} \:\cos(2 \:\pi \:u_2)$}
      \Let{$z_2$}{$\sqrt{\smash[b]{-2 \:ln(n_1)}} \:\sin(2 \:\pi \:u_2)$}
      \Statex
    \EndFunction
  \end{algorithmic}
\end{algorithm}

\vspace{5mm}
Para todas as distribuições, o passo do caminhante é o valor gerado pelo gerador
de números pseudoaleatório.

\vspace{5mm}
Cada simulação gera 2 grupos de 10 caminhadas aleatórias, com 10.000 passos,
para cada distribuição. Cada elemento do par gera um grupo distinto de
caminhadas.
