
\section{Conclusões}

A raiz da média quadrática para o caminhante q-Gaussiano, com valor $q > 5/3$,
parece violar o envelope do caminhante aleatório, caracterizado por
$f(t) = t^{1/2}$.

\vspace{5mm}
Para que possamos estimar o expoente de difusão e validá-lo com a estimativa
presente na literatura, se mostrou necessário simular um maior número de
caminhadas. Isso se dá pelo fato de que há um número considerável de saltos
abruptos na curva da raiz da média quadrática, que dificulta o ajuste.

\vspace{5mm}
Caso o expoente de difusão seja superior a $1/2$, a Caminhada Aleatória com
passo q-Gaussiano poderá simular comportamentos superdifusivos, que se enquadra
como difusão anômala.
